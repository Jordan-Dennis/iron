\documentclass[a4paper]{article}
\hoffset=-1in
\voffset=-1in
\textwidth=175mm
\textheight=250mm
\usepackage{amsmath}
\usepackage{graphicx}
\usepackage{multicol}
\def\citation{{\bfseries (Annon, dd/mm/yyyy)}}
\begin{document}
    \begin{multicols}{2}
    \section*{Introduction}
        The study of magnetic materials is an area of academic %
        and industrial interest \citation. For example, magnetic %
        technologies are important in the ongoing development of %
        quantum computers, superconducting circuits and other %
        examples in electronics \citation. At a fundamental level %
        magnetisation is a well understood phenomenon, yet it is %
        difficult to theoretically model. One simple model of magnetic %
        materials is the Ising model. 


        The Ising model is the simplest model of a ferro-magnet \citation. %
        Despite the simplicity of the Ising model it displays rich %
        physical behaviour and has analytic solutions in one and %
        two dimensions \citation.  The Ising model is the simplest model %
        to account for inter-molecular interactions and contain a phase %
        transition. This makes it an excellent medium for studying %
        magnetic phenomenon \citation. 


        By modifying the basic Ising model we can simulate many %
        phenomenon including glasses \citation. The Ising model %
        has broader significance and can be used to construct very %
        simple neural networks called Boltzmann machines \citation. We tested %
        one and two dimension Ising models and confirmed that they %
        matched theoretical predictions.


    \section*{Theory}
        Materials have internal interactions. As physicists we like %
        to ignore these where possible but often these approximations %
        limit the accuracies of our models \citation. Magnetic %
        phenomenon are no different. To understand how spins interact %
        in a magnet it helps to first construct the simplest possible %
        model without interactions; a para-magnet.


        Consider our magnet as a one-dimensional chain of atomic spins. %
        For the moment ignore any external magnetic field and just %
        consider the spins in isolation. Now lets limit the spins to %
        be fixed up or down along one axis. If there are no interactions %
        between the spins the energy is fixed. If we add an external %
        magnetic field then we would expect the ensemble to develop a %
        net magnetisation.


        If the system has thermal energy we would expect some of the %
        spins to align themselves anti-parallel to the magnetic field. %
        We can see this affect by considering the partition function %
        for a single spin in the ensemble. If the spin is aligned with %
        the magnetic field then the energy is \(-sB\), where \(s\) is %
        the unit of magnetisation carried by the single spin and \(B\) %
        is the strength of the external magnetic field. If the spin is %
        anti-aligned with the field then the energy is \(sB\). 


        This is a simple two level system and the partition function %
        is given by, 
        %
        \begin{align}
            Z &= \sum_{s = \pm 1}\exp\left(-\frac{sB}{\tau}\right)\nonumber\\
                &= \exp\left(-\frac{sB}{\tau}\right) + 
                    \exp\left(\frac{sB}{\tau}\right)\nonumber\\
                &= 2\cosh\left(\frac{sB}{\tau}\right),
            \label{eqn:1}
        \end{align}
        %
        where, \(\tau = kT\) is the temperature in units of energy. %
        The probability of the spins being anti-aligned with the %
        field is therefore, 
        %
        \begin{align}
            P &= \frac{\exp\left(-\frac{sB}{\tau}\right)}
                    {2\cosh\left(\frac{sB}{\tau}\right)}.
            \label{eqn:2}
        \end{align}
        %
        Hence, as the temperature increase we expect the number of %
        anti-aligned spins to increase and as we increase the %
        magnetic field we expect the number of anti-aligned spins %
        to decrease. 


        Since each of the spins in a para-magnetic system is independent %
        the partition function of an ensemble of \(N\) spins is just %
        the product of \(N\) partition functions for the single spin %
        case. However, since the spins are indistinguishable we must %
        also divide by a Gibbs correction factor of \(N!\). %
        The probability of finding a particular state however, %
        is a case that is worth studying, since it indicates a %
        diveregence between the Ising model of a ferro-magnet and %
        a para-magnet in a magnetic field. First we need to define %
        our state. 


        % TODO: These need fixing to include the gibbs correction 
        % factor. 
        The energy of the system, and any other physical %
        parameters, only depend on the number of spins that are %
        aligned with the magnetic field and not specifically %
        which spins are aligned with the field. Naively we might %
        expect that the probability of having \(N_{\uparrow}\) %
        spins aligned with the field would be,
        %
        \begin{align}
            P(N_{\uparrow}) &= \frac{
                    \exp\left(-\frac{sN_{\uparrow}B}{\tau}\right)
                    \exp\left(\frac{s(N - N_{\uparrow})B}{\tau}\right)}{
                    \cosh\left(\frac{sB}{\tau}\right)^{N}}.
            \label{eqn:3}
        \end{align}
        %
        However, equation \ref{eqn:3} has failed to account for the %
        multiple micro-states that occupy this macro-state. We can %
        account for this by multiplying by the multiplicity, which %
        can be found using the chose function,
        %
        \begin{align}
            P(N_{\uparrow}) &= \frac{N!}{N_{\uparrow}!(N - N_{\uparrow})!}
                    \frac{\exp\left(-\frac{sN_{\uparrow}B}{\tau}\right)
                    \exp\left(\frac{s(N - N_{\uparrow})B}{\tau}\right)}{
                    \cosh\left(\frac{sB}{\tau}\right)^{N}}.
            \label{eqn:4}
        \end{align}
        %
        Equation \ref{eqn:4} is the correct expression for the probability.


        % TODO: I need to complete the goals that I outline in this 
        % paragraph. 
        We can also determine the energy of the system and the entropy. %
        Using the free energy we can also find the equilibrium state and %
        hence determine the equilibrium properties of the system as a %
        function of the temperature.


        % TODO: For no external magnetic field is there a theoretical 
        % derivation that I can do for the anti-ferromagnetic case?


        Para-magnets are a useful toy model but from our experience %
        with natural and manufactured magnets we know that it is %
        possibe to construct systems that are magnetic without external %
        fields. The one-dimensional Ising model is a simple model of %
        such systems. The Ising model is a natural extension of the %
        paramagnetic model that we discussed, and operates on the same %
        spin lattice. 


        The Ising model differs because it adds very simple interactions %
        between neighbouring spins. This interaction favours pairs that %
        are aligned by reducing the energy of this scenario. Representing %
        up spins as \(+1\) and down spins as \(-1\) we can represent this %
        mutal interaction as \(\Delta \epsilon = \varepsilon s_{i}s_{i + 1}\), %
        where \(\Delta \epsilon\) is the energy contribution of the %
        interaction, \(\varepsilon\) is a scaling factor that represents %
        the strength of the interaction and \(s_{i}\) is the \(i^{th}\) %
        spin in the chain. 


        % TODO: Add the case where there is an external magnetic field and 
        % see if I can solve it. This will be cool to do. 
        An obvious way that this differs from the para-magnetic model %
        is that each spin can be independently resolved, and hence %
        the Gibbs correction function is no longer needed. Let's isolate %
        the case we are considering from a magnetic field as this simplifies %
        the calculations. Again we start be considering the partition %
        function of an individual pair. Similarly to the para-magnetic case %
        this is a two level system; either the pair are aligned or they %
        are anti-aligned with the corresponding energies.
        %
        \begin{align}
            Z_{i} &= \sum_{s_{i} = \pm 1}
                    \exp\left(-\frac{\varepsilon s_{i}s_{i+1}}{\tau}\right)
                    \nonumber\\
                &= \exp\left(-\frac{\varepsilon}{\tau}\right) +
                    \exp\left(\frac{\varepsilon}{\tau}\right)
                    \nonumber\\
                &= 2\cosh\left(\frac{\varepsilon}{\tau}\right).
           \label{eqn:5}
        \end{align}
        %
        It is worth noting the strong similarity between equations %
        \ref{eqn:5} and \ref{eqn:1}. 


        Similarly to the para-magnetic case we can multiply the system %
        partition functions of single constituents together to get the %
        partition function of the entire system. However, the condition %
        to do this was that the constiutuents were independent, but the %
        Ising model contains interactions. In the case of the Ising model %
        the constituents that are independent are the pairs, not the %
        individual spins. You may think think then that we only consider %
        \(N / 2\) unique pairs but this is not the case. In a chain each %
        spin is counted in two pairs so the power is still \(N\). 


        A small detail that I skipped was what happens at the boundary. %
        The two spins on the end of the chains are not (neccessarily) %
        counted twice. In the limit of a very large chain of spins we %
        can see that the boundary affect will not matter however, we %
        got about this nuance in a much more interesting way by considering %
        cyclic boundary conditions. That is to say that the spin on the %
        far end of the chain is a neighbour to the spin at the start of the %
        chain and vice versa. 


        Given the partition function \(Z = (2\cosh(\varepsilon / \tau))^{N}\), we 
        calculated the internal energy using,
        %
        \begin{align}
            U &= \tau^{2}\partial_{\tau}\ln(Z)\label{EQN1}\\
                &= \tau^{2}\partial_{\tau}
                    \ln\left(2\cosh\left(\frac{\varepsilon}{\tau}\right)^{N}\right)
                    \nonumber \\
                &= N\tau^{2}\partial_{\tau}
                    \ln\left(2\cosh\left(\frac{\varepsilon}{\tau}\right)\right)
                    \nonumber \\
                &= N\tau^{2}\partial_{\tau}
                    \left(2\cosh\left(\frac{\varepsilon}{\tau}\right)\right)
                    \frac{1}{2\cosh\left(\frac{\varepsilon}{\tau}\right)}
                    \nonumber \\
                &= N\tau^{2}\partial_{\tau}\left(\frac{\varepsilon}{\tau}\right)
                    \frac{\sinh\left(\frac{\varepsilon}{\tau}\right)}
                    {\cosh\left(\frac{\varepsilon}{\tau}\right)}\nonumber \\
                &= -\varepsilon N\tanh\left(\frac{\varepsilon}{\tau}\right)
                    \label{EQN2}.
        \end{align}
        %
        We calculated the free energy of the system using,
        %
        \begin{align}
            F &= -\tau\ln Z \label{EQN3}\\
                &= -\tau\ln\left(\left(
                    2\cosh\left(\frac{\varepsilon}{\tau}\right)\right)^{N}\right) 
                    \nonumber \\
                &= -N\tau\ln\left(
                    2\cosh\left(\frac{\varepsilon}{\tau}\right)\right) 
                    \nonumber \\
                &= -N\tau\ln\left(
                    \exp\left(\frac{\varepsilon}{\tau}\right) + 
                    \exp\left(-\frac{\varepsilon}{\tau}\right)\right) \nonumber \\
                &= -N\tau\ln\left(\exp\left(\frac{\varepsilon}{\tau}\right)
                    \left(1 + \exp\left(-2\frac{\varepsilon}{\tau}\right)\right)
                    \right)\nonumber \\
                &= -N\tau\ln\left(\exp\left(\frac{\varepsilon}{\tau}\right)\right)
                    - N\tau\ln\left(1 + 
                    \exp\left(-2\frac{\varepsilon}{\tau}\right)\right) \nonumber \\
                &= -N\varepsilon - N\tau\ln\left(1 + 
                    \exp\left(-2\frac{\varepsilon}{\tau}\right)\right)\label{EQN4}. 
        \end{align}
        %
        The entropy followed from the combination of Equation \ref{EQN4} and %
        Equation \ref{EQN2} using Equation \ref{EQN5},
        %
        \begin{align}
            \tau\sigma &= F - U \label{EQN5} \\
                &= -N\varepsilon\tanh\left(\frac{\varepsilon}{\tau}\right) + 
                    N\varepsilon + N\tau\ln\left(1 + 
                    \exp\left(-2\frac{\varepsilon}{\tau}\right)\right)\nonumber \\
            \sigma &= \frac{\varepsilon}{\tau}\left(1 - 
                    \tanh\left(\frac{\varepsilon}{\tau}\right)\right) +
                    \ln\left(1 + \exp\left(-2\frac{\varepsilon}{\tau}\right)\right)
                    \label{EQN6}.
        \end{align} 
        %
        Finally, we determined the specific heat using Equation \ref{EQN7} %
        and Equation \ref{EQN2},
        %
        \begin{align}
            C &= \partial_{\tau}U\label{EQN7}\\
                &= \partial_{\tau}\left(-N\varepsilon\tanh\left(
                    \frac{\varepsilon}{\tau}\right)\right)\nonumber\\
                &= -N\varepsilon\partial_{\tau}\left(\frac{\varepsilon}{\tau}\right)
                    \frac{1}{\cosh^{2}\left(\frac{\varepsilon}{\tau}\right)}
                    \nonumber\\
                &= \frac{N\varepsilon^{2}}{\tau^{2}\cosh^{2}\left(
                    \frac{\varepsilon}{\tau}\right)}\label{EQN8}.
        \end{align}    
        

        :x
        We have spent a lot of time discussing the one-dimensional scenario %
        but real systems are typically higher dimensional. There is an %

    \section*{Method}
        
    \end{multicols}
\end{document}
